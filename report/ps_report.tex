\documentclass[letterpaper,12pt]{article}

\usepackage{tabularx} % extra features for tabular environment
\usepackage{amsmath}  % improve math presentation
\usepackage{amssymb}
\usepackage{multirow}
\usepackage{xcolor}
\usepackage{gensymb}
\usepackage{appendix}
\usepackage{verbatim}
\usepackage{bigints}

\usepackage{mathtools}
\usepackage{gensymb}
\usepackage{float}
\usepackage{listings}
\usepackage[export]{adjustbox}
\usepackage[super]{nth}
\usepackage{graphicx} % takes care of graphic including machinery
\usepackage[margin=1in,letterpaper]{geometry} % decreases margins
\usepackage{cite} % takes care of citations
\usepackage[final]{hyperref} % adds hyper links inside the generated pdf file

\newcommand*{\tran}{^{\mkern-1.5mu\mathsf{T}}}
\DeclarePairedDelimiter\ceil{\lceil}{\rceil}
\DeclarePairedDelimiter\floor{\lfloor}{\rfloor}
\hypersetup{
    colorlinks=false,       % false: boxed links; true: colored links
    linkcolor=blue,        % color of internal links
    citecolor=blue,        % color of links to bibliography
    filecolor=magenta,     % color of file links
    urlcolor=blue         
}
%++++++++++++++++++++++++++++++++++++++++++++++++++++++++++++++++++++++++++++++++



%++++++++++++++++++++++++++++++++++++++++++++++++++++++++++++++++++++++++++++++++
% Start modifying the labwork number, your team number and the name and METU id
% of your group members.
\newcommand{\reporttitle}{Term Project}
\newcommand{\reportauthor}{ Volkan Aydıngül (Id: 0075359 )\\
                            }
                            % If any teammate does not help to write this report,
                            % you may not write his/her name here.
%++++++++++++++++++++++++++++++++++++++++++++++++++++++++++++++++++++++++++++++++



%++++++++++++++++++++++++++++++++++++++++++++++++++++++++++++++++++++++++++++++++
% DO NOT MODIFY THIS SECTION
\begin{document}
\begin{titlepage}
\newcommand{\HRule}{\rule{0.7\linewidth}{0.5mm}}
\begin{center} % Center remainder of the page
%	LOGO SECTION
\includegraphics[width = 8cm]{figures/koc_logo.png}

%	HEADING SECTIONS
\textsc{\Large PHYS 514 - Computational Physics}\\[1.5cm] 
%	TITLE SECTION
\HRule \\[0.6cm]
{ \huge \bfseries \reporttitle}\\ % Title of your document
\HRule \\[1.5cm]
\end{center}
\vspace{2cm}
%	AUTHOR SECTION
\begin{flushleft} \large
\textit{Author:}\\
\reportauthor% Your name
\end{flushleft}
\vspace{2cm}
\makeatletter
Date: \@date 
\vfill % Fill the rest of the page with whitespace
\makeatother
\end{titlepage}
%++++++++++++++++++++++++++++++++++++++++++++++++++++++++++++++++++++++++++++++++




\tableofcontents
\newpage





%\begin{figure}[H] 
%   \centering \includegraphics[width=\columnwidth]{figures/figure.png}           
%                \caption{Caption}                
%                   \label{fig:label}
%   \end{figure}
\section{Derivation of \textit{Lane-Emden Equation}}

\paragraph{} In this project, the first task is to derive the \textit{Lane-Emden} equation to be able to have a general expression for the \textit{hydrostatic equilibrium}. In the beginning, it is known that:

\begin{equation}
    \label{eq:dmdr}
    \frac{dm}{dr} = 4\pi r^2\rho
\end{equation}

\begin{equation}
    \label{eq:dpdr}
    \frac{dp}{dr} = -\frac{Gm\rho}{r^2}
\end{equation}

where $m$, $\rho$, and $p$ are functions of $r$. To be able to solve this system of differential equations, the expression for the $\rho$ is required, which is not explicitly given. However, by using the \textit{equation of state (EOS)}, one can easily construct a relation between $p$ and $\rho$, such that:

\begin{equation*}
    p = \frac{k_B}{\mu m_H}T\rho
\end{equation*}

Assuming that isentropic relation holds, one can conclude that:

\begin{equation}
    \label{eq:istro}
    p = K\rho^\gamma=K\rho^{1 + \frac{1}{n}} 
\end{equation}

\paragraph{} To be able to obtain a unified expression, one can inspect the \textit{mass} variable in the \eqref{eq:dmdr}, and \eqref{eq:dpdr}. It is required to calculate the derivative of the \eqref{eq:dpdr} to write \eqref{eq:dpdr} in terms of \eqref{eq:dmdr}. Differentiating \eqref{eq:dpdr} yields:

\begin{equation}
    \label{eq:d2pdr2-1}
    \frac{d^2p}{dr^2} = \frac{2Gm\rho}{r^3} + \frac{-G\rho}{r^2}\frac{dm}{dr} + \frac{-Gm}{r^2}\frac{d\rho}{dr}
\end{equation}

\paragraph{} Before further processing on the \eqref{eq:d2pdr2-1}, the some terms must be revised, that is, \eqref{eq:d2pdr2-1} can be rewritten in the following form:

\begin{equation}
    \label{eq:d2pdr2-2}
    \frac{d^2p}{dr^2} = \frac{2}{r}\left(-\frac{dp}{dr}\right) + \frac{d\rho}{dr}\left(\frac{1}{\rho}\frac{dp}{dr}\right) + 4\pi G \rho^2
\end{equation}

\paragraph{}Multiplying \eqref{eq:d2pdr2-2} with $\frac{r^2}{\rho}$ and rearranging the terms yields:

\begin{equation}
    \label{eq:totder-1}
    \frac{r^2}{\rho}\frac{d^2p}{dr^2} + 2r\left(\frac{1}{\rho}\frac{dp}{dr}\right) - r^2\frac{d\rho}{dr}\left(\frac{1}{\rho^2}\frac{dp}{dr}\right) = 4\pi G r^2 \rho
\end{equation}

\paragraph{}Observing \eqref{eq:totder-1}, one can easily realize that the left hand side is nothing but the total derivative of the term $\frac{r^2}{\rho}\frac{dp}{dr}$. Hence, in conclusion, the \eqref{eq:d2pdr2-1}, can be written as the following:

\begin{equation}
    \label{eq:totder-2}
    \frac{d}{dr}\left(\frac{r^2}{\rho}\frac{dp}{dr}\right) = 4\pi G r^2 \rho
\end{equation}

\paragraph{} From now on, the finite amount of transformation should be applied on the $\rho$ and $r$. Firstly, assume that the following relations hold:

\begin{equation*}
    %\theta = \rho^{\frac{1}{n}} \rightarrow \rho = \theta^n
    \rho = \rho_c \theta^n
\end{equation*}

Therefore, 

\begin{equation}
    \label{eq:dp}
    p = K \rho_c^{1+\frac{1}{n}} \theta^{n+1} \rightarrow dp = K \rho_c^{1+\frac{1}{n}} \left(n+1\right)\theta^n d\theta
\end{equation}

\paragraph{} Applying \eqref{eq:dp} to \eqref{eq:totder-2}, one can write:

\begin{equation}
    \frac{d}{dr}\left(\frac{r^2}{\rho_c \theta^n} K \rho_c^{1+\frac{1}{n}} \left(n+1\right)\theta^n \frac{d\theta}{dr}\right) = 4\pi G r^2 \rho_c \theta^n \rightarrow \frac{1}{r^2}\frac{d}{dr}\left(r^2 \frac{K \rho_c^{\frac{1}{n}-1} \left(n+1\right)}{4 \pi G} \frac{d\theta}{dr}\right) + \theta^n = 0
\end{equation}

\paragraph{} Followingly, another transformation that should be applied might be:

\begin{equation*}
    \alpha^2 = \frac{K \rho_c^{\frac{1}{n}-1} \left(n+1\right)}{4 \pi G} \rightarrow r = \alpha \xi
\end{equation*}

\paragraph{} Then, \eqref{eq:totder-2} can be written as:

\begin{equation*}
    \left(\frac{1}{\alpha^2}\frac{1}{\xi^2}\right)\frac{1}{\alpha}\frac{d}{d\xi}\left(\alpha^2\xi^2\alpha^2\frac{1}{\alpha}\frac{d\theta}{d\xi}\right) + \theta^n = 0
\end{equation*}

\begin{equation}
    \label{eq:final}
   \frac{1}{\xi^2}\frac{d}{d\xi}\left(\xi^2\frac{d\theta}{d\xi}\right) + \theta^n = 0
\end{equation}

\paragraph{} The approximate solution of the \eqref{eq:final} via \textit{Wolfram Mathematica} reads:

\begin{equation*}  
      \theta(\xi) =  1-\frac{\xi ^2}{6}+\frac{n \xi ^4}{120}+\frac{n (5-8 n) \xi ^6}{15120}+\dots
\end{equation*}

\paragraph{} For $n = 1$, solution can be written as:

\begin{equation*}
    \theta(\xi) =  \frac{\sin (\xi )}{\xi }
\end{equation*}

\subsection{Calculation of Total Mass}

\paragraph{} To be able to calculate the mass of the star for a given radius, one can easily make use of the \eqref{eq:dmdr}, that is, integration of the \eqref{eq:dmdr} for a certain radius interval yields the mass of the subject star. That being said, the following relation can be observed:

\begin{equation*}
   M =  \int_0^R {4\pi r^2 \rho(r)}dr
\end{equation*}

After the change of variables,

\begin{equation*}
    M = \int_0^\Xi {4 \pi \alpha^2 \xi^2 \rho_c \theta^n (\xi)} \alpha d\xi
\end{equation*}
where $\Xi = \frac{R}{\alpha}$
\begin{equation*}
    M =4 \pi \rho_c \alpha^3 \int_0^\Xi {\xi^2 \theta^n (\xi)} d\xi
\end{equation*}

\paragraph{} By using the relation in \eqref{eq:final}, one can deduce that:

\begin{equation*}
    M =4 \pi \rho_c \alpha^3 \int_0^\Xi {\xi^2 \frac{1}{\xi^2}\frac{d}{d\xi}\left(\xi^2\frac{d\theta}{d\xi}\right) } d\xi
\end{equation*}

\begin{equation*}
    M = 4 \pi \rho_c \alpha^3 \Xi^2 \theta' = 4 \pi \rho_c \alpha^3 \Xi^3 \frac{\theta'}{\Xi}
\end{equation*}
\paragraph{} Finally, the total mass can be read as:
\begin{equation*}
\boxed{ M = 4 \pi \rho_c R^3 \frac{\theta'}{\Xi}}
\end{equation*}

\end{document}



              


